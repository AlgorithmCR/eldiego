% LaTeXar con :
%  pdflatex eldiego.tex
%"The PDF file may contain up to 25 pages of reference material, single-sided, letter or A4 size, with text and illustrations readable by a person with correctable eyesight without magnification from a distance of 1/2 meter."
\documentclass[10pt,landscape,twocolumn,a4paper,notitlepage]{article}
\usepackage{hyperref}
\usepackage[spanish, activeacute]{babel}
\usepackage[utf8]{inputenc}
\usepackage{fancyhdr}
\usepackage{lastpage}
\usepackage{listings}
\usepackage{amssymb}
\usepackage[usenames,dvipsnames]{color}
\usepackage{graphicx}
\usepackage{wrapfig}
\usepackage{amsmath}
\usepackage{makeidx}

%%% Espacio de los titulos
\usepackage{titlesec} 
\titleformat{\section} {\normalfont\Large\bfseries\centering}{\thesection}{1em}{}
\titleformat{\subsection} {\normalfont\large\bfseries\centering}{\thesubsection}{1em}{}
\titlespacing*{\subsection}{0pt}{0pt}{0pt} % interespacio de las subsecciones
  
%%% Margenes
\setlength{\columnsep}{0.0in}    % default=10pt
\setlength{\columnseprule}{0.0pt}    % default=0pt (no line)
 
\addtolength{\textheight}{2.9in}
\addtolength{\topmargin}{-1.2in}     % ~ -0.5 del incremento anterior
 
\addtolength{\textwidth}{1.8in}
\addtolength{\oddsidemargin}{-0.9in} % -0.5 del incremento anterior
 
\setlength{\headsep}{0.08in}
\setlength{\parskip}{0in}
\setlength{\headheight}{15pt}
\setlength{\parindent}{0mm}
 
%%% Encabezado y pie de pagina
\pagestyle{fancy}
\fancyhead[LO]{\textbf{\title}}
\fancyhead[C]{\leftmark\ -\ \rightmark}
\fancyhead[RO]{P\'agina \thepage\ de \pageref{LastPage}}
\renewcommand{\headrulewidth}{0.4pt}
\fancyfoot{}
\definecolor{darkblue}{rgb}{0,0,0.4}
%%% Configuracion de Listings
\lstloadlanguages{C++}
\lstnewenvironment{code}
	{%\lstset{	numbers=none, frame=lines, basicstyle=\small\ttfamily, }%
	 \csname lst@SetFirstLabel\endcsname}
	{\csname lst@SaveFirstLabel\endcsname}
\lstset{% general command to set parameter(s)
	language=C++, basicstyle=\small\ttfamily, keywordstyle=\slshape,
	emph=[1]{tipo,usa}, emphstyle={[1]\sffamily\bfseries},
	morekeywords={tint,forn,forsn},
	basewidth={0.47em,0.40em},
	columns=fixed, fontadjust, resetmargins, xrightmargin=5pt, xleftmargin=15pt,
	flexiblecolumns=false, tabsize=2, breaklines,	breakatwhitespace=false, extendedchars=true,
	numbers=left, numberstyle=\tiny, stepnumber=1, numbersep=9pt,
	frame=l, framesep=3pt,
    basicstyle=\ttfamily,
    keywordstyle=\color{darkblue}\ttfamily,
    stringstyle=\color{magenta}\ttfamily,
    commentstyle=\color{RedOrange}\ttfamily,
    morecomment=[l][\color{OliveGreen}]{\#}
}

\lstdefinestyle{C++}{
	language=C++, basicstyle=\small\ttfamily, keywordstyle=\slshape,
	emph=[1]{tipo,usa,tipo2}, emphstyle={[1]\sffamily\bfseries},
	morekeywords={tint,forn,forsn},
	basewidth={0.47em,0.40em},
	columns=fixed, fontadjust, resetmargins, xrightmargin=5pt, xleftmargin=15pt,
	flexiblecolumns=false, tabsize=2, breaklines,	breakatwhitespace=false, extendedchars=true,
	numbers=left, numberstyle=\tiny, stepnumber=1, numbersep=9pt,
	frame=l, framesep=3pt,
    basicstyle=\ttfamily,
    keywordstyle=\color{darkblue}\ttfamily,
    stringstyle=\color{magenta}\ttfamily,
    commentstyle=\color{RedOrange}\ttfamily,
    morecomment=[l][\color{OliveGreen}]{\#}
}
 
%%% Macros
\def\nbtitle#1{\begin{Large}\begin{center}\textbf{#1}\end{center}\end{Large}}
\def\nbsection#1{\section{#1}}
\def\nbsubsection#1{\subsection{#1}}
\def\nbcoment#1{\begin{small}\textbf{#1}\end{small}}
\newcommand{\comb}[2]{\left( \begin{array}{c} #1 \\ #2 \end{array}\right)}
\def\complexity#1{\texorpdfstring{$\mathcal{O}(#1)$}{O(#1)}}
 \newcommand\cppfile[2][]{
\lstinputlisting[style=C++,linerange={#1}]{#2}
}

\begin{document}
\def\title{Caloventor en Dos - Universidad Nacional de Rosario}
\section{algorithm}%%%%%%%%%%%%%%%%%%ALGORITHM%%%%%%%%%%%%%%%%%%
\textbf{\#include $<$algorithm$>$ \#include $<$numeric$>$ \\}
\begin{tabular}{|l|l|p{5.4cm}|} \hline
\textbf{Algo} & \textbf{Params} &  \textbf{Funcion} \\  \hline
%swap & e1, e2 &  da vuelta e1,e2 & $1$\\\hline
sort, stable\_sort & f, l &  ordena el intervalo \\  \hline
%is\_sorted & f, l &  \textit{bool} si esta ordenado \\  \hline
nth\_element & f, nth, l & \textit{void} ordena el n-esimo, y \\ && particiona el resto \\  \hline
fill, fill\_n & f, l / n, elem & \textit{void} llena [f, l) o [f, \\ && f+n) con elem \\  \hline
lower\_bound, upper\_bound & f, l, elem & \textit{it} al primer / ultimo donde se \\ && puede insertar elem para que\\ && quede ordenada \\  \hline
binary\_search & f, l, elem & \textit{bool} esta elem en [f, l) \\  \hline
copy & f, l, resul & hace resul+$i$=f+$i$ $\forall i$ \\  \hline
find, find\_if, find\_first\_of & f, l, elem & \textit{it} encuentra i $\in$[f,l) tq. i$=$elem, \\ & / pred / f2, l2 & pred(i), i$\in$[f2,l2)\\\hline
count, count\_if & f, l, elem/pred & cuenta elem, pred(i)\\\hline
search & f, l, f2, l2 & busca [f2,l2) $\in$ [f,l)\\\hline
replace, replace\_if & f, l, old & cambia old / pred(i) por new \\ & / pred, new &\\\hline
reverse & f, l & da vuelta\\\hline
partition, stable\_partition & f, l, pred & pred(i) ad, !pred(i) atras\\\hline
%min, max & e1, e2 & men / may & $1$\\\hline
min\_element, max\_element & f, l, [comp] & \textit{it} min, max de [f,l]\\\hline
lexicographical\_compare & f1,l1,f2,l2 & \textit{bool} con [f1,l1]<[f2,l2]\\\hline
next/prev\_permutation & f,l & deja en [f,l) la perm sig, ant\\\hline
set\_intersection, & f1, l1, f2, l2, res & [res, $\ldots$) la op. de conj\\
set\_difference, set\_union, & & \\
set\_symmetric\_difference, & &\\\hline
push\_heap, pop\_heap, & f, l, e / e / & mete/saca e en heap [f,l), \\
make\_heap & & hace un heap de [f,l)\\\hline
is\_heap & f,l & \textit{bool} es [f,l) un heap\\\hline
accumulate & f,l,i,[op] & \textit{T} $=$ $\sum$/oper de [f,l)\\\hline
inner\_product & f1, l1, f2, i & \textit{T} $=$ i $+$ [f1, l1) . [f2, $\ldots$ )\\\hline
partial\_sum & f, l, r, [op] & r+i = $\sum$/oper de [f,f+i] $\forall i \in$[f,l)\\\hline
%power & e, i, op & \textit{T} = $e^{n}$\\\hline
\_\_builtin\_ffs& unsigned int & Pos. del primer 1 desde la derecha\\\hline
\_\_builtin\_clz & unsigned int & Cant. de ceros desde la izquierda.\\\hline
\_\_builtin\_ctz & unsigned int & Cant. de ceros desde la derecha.\\\hline
\_\_builtin\_popcount & unsigned int & Cant. de 1’s en x.\\\hline
\_\_builtin\_parity & unsigned int & 1 si x es par, 0 si es impar.\\\hline
\_\_builtin\_llXXXXXX & unsigned ll & = pero para long long's.\\\hline
\end{tabular}


\section{Estructuras}%%%%%%%%%%%%%%%%%%ESTRUCTURAS%%%%%%%%%%%%%%%%%%
\subsection{RMQ (static)}
Dado un arreglo y una operacion asociativa \emph{idempotente}, get(i, j) opera sobre el rango [i, j). Restriccion: LVL $\ge$ ceil(logn); Usar [ ] para llenar arreglo y luego build().
\cppfile{estructuras/rmq.static.cpp}
\subsection{RMQ (dynamic)}
\cppfile{estructuras/rmq.dynamic.cpp}
\subsection{RMQ (lazy)}
\cppfile{estructuras/rmq.lazy.cpp}
\subsection{RMQ (persistente)}
\cppfile[23-53]{estructuras/rmq.persistent.cpp}
\subsection{Fenwick Tree}
\cppfile{estructuras/fenwick.cpp}
\subsection{Union Find}
\cppfile{grafos/union.find.cpp}
\subsection{Disjoint Intervals}
\cppfile{estructuras/disjoint.intervals.cpp}
\subsection{RMQ (2D)}
\cppfile{estructuras/rmq.2d.cpp}
\subsection{Big Int}
\cppfile{estructuras/bigint.villa.cpp}
\subsection{HashTables}
\cppfile[16-30]{hash.cpp}
\subsection{Modnum}
\cppfile{estructuras/mnum.cpp}
\subsection{Treap para set}
\cppfile[16-89]{estructuras/treap.cpp}
\subsection{Treap para arreglo}
\cppfile[17-87]{estructuras/treaparr.cpp}
\subsection{Convex Hull Trick}
\cppfile[18-55]{estructuras/convexhull.trick.cpp}
\subsection{Convex Hull Trick (Dynamic)}
\cppfile[17-56]{estructuras/convexhull.trick.dyn.cpp}
%\subsection{Gain-Cost Set}
%\cppfile[20-43]{estructuras/gain-cost.set.cpp}
\subsection{Set con busq binaria}
\cppfile[17-26]{estructuras/order.tree.cpp}


\section{Algos}%%%%%%%%%%%%%%%%%%ALGORITMOS%%%%%%%%%%%%%%%%%%%%%%%%%%
\subsection{Longest Increasing Subsecuence}
\cppfile[14-42]{algos/lis.cpp}
\subsection{Alpha-Beta prunning}
\cppfile{algos/alphabeta.cpp}
\subsection{Mo's algorithm}
\cppfile{algos/mosalgorithm.cpp}


\section{Strings}%%%%%%%%%%%%%%%%%%STRINGS%%%%%%%%%%%%%%%%%%%%%%%%%%
\subsection{Manacher}
\cppfile[18-37]{string/manacher.cpp}
\subsection{KMP}
\cppfile[21-38]{string/kmp.cpp}
\subsection{Trie}
\cppfile{string/trie.cpp}
%\subsection{Suffix Array (corto, nlog2n)}
%\cppfile[12-26]{string/suffix.array.short.cpp}
\subsection{Suffix Array (largo, nlogn)}
\cppfile[-34]{string/suffix.array.cpp}
\subsection{String Matching With Suffix Array}
\cppfile[37-58]{string/suffix.array.cpp}
\subsection{LCP (Longest Common Prefix)}
\cppfile[60-75]{string/suffix.array.cpp}
\subsection{Corasick}
\cppfile[9-45]{string/corasick.cpp}
\subsection{Suffix Automaton}
\cppfile[16-61]{string/suffix.automaton.cpp}
\subsection{Z Function}
\cppfile[17-27]{string/zfunction.cpp}


\section{Geometria}%%%%%%%%%%%%%%%%%%GEOMETRIA%%%%%%%%%%%%%%%%%%%%%%
\subsection{Punto}
\cppfile[2-33]{geometria/pto.cpp}
\subsection{Orden radial de puntos}
\cppfile{geometria/orden.radial.cpp}
\subsection{Line}
\cppfile{geometria/line.cpp}
\subsection{Segment}
\cppfile{geometria/segm.cpp}
%\subsection{Rectangle}
%n\cppfile{geometria/rect.cpp}
\subsection{Polygon Area}
\cppfile{geometria/area.cpp}
\subsection{Circle}
\cppfile{geometria/circle.cpp}
\subsection{Point in Poly}
\cppfile{geometria/point.in.poly.cpp}
\subsection{Point in Convex Poly log(n)}
\cppfile{geometria/point.in.convex.poly.cpp}
%\subsection{Convex Check CHECK}
%\cppfile{geometria/convex.check.cpp}
\subsection{Convex Hull}
\cppfile{geometria/convex.hull.cpp}
\subsection{Cut Polygon}
\cppfile{geometria/cut.polygon.cpp}
\subsection{Bresenham}
\cppfile{geometria/bresenham.cpp}
%\subsection{Rotate Matrix}
%\cppfile{geometria/rotate.cpp}
\subsection{Interseccion de Circulos en n3log(n)}
\cppfile{geometria/int.circs.cpp}


\section{Math}%%%%%%%%%%%%%%%%%%MATH%%%%%%%%%%%%%%%%%%%%%%%%%%%%%%%%
\subsection{Identidades}
{
$\sum_{i=0}^n\binom{n}{i}=2^n$

$\sum_{i=0}^n i\binom{n}{i}=n*2^{n-1}$

$\sum_{i=m}^n i = \frac{n(n+1)}{2} - \frac{m(m-1)}{2} = \frac{(n+1-m)(n+m)}{2}$

$\sum_{i=0}^n i = \sum_{i=1}^n i = \frac{n(n+1)}{2}$

$\sum_{i=0}^n i^2 = \frac{n(n+1)(2n+1)}{6} = \frac{n^3}{3} + \frac{n^2}{2} + \frac{n}{6}$

$\sum_{i=0}^n i(i-1) = \frac{8}{6}(\frac{n}{2})(\frac{n}{2}+1)(n+1)$ (doubles) $\rightarrow$ Sino ver caso impar y par

$\sum_{i=0}^n i^3 = \left(\frac{n(n+1)}{2}\right)^2 = \frac{n^4}{4} + \frac{n^3}{2} + \frac{n^2}{4} = \left[\sum_{i=1}^n i\right]^2$

$\sum_{i=0}^n i^4 = \frac{n(n+1)(2n+1)(3n^2+3n-1)}{30} = \frac{n^5}{5} + \frac{n^4}{2} + \frac{n^3}{3} - \frac{n}{30}$

$\sum_{i=0}^n i^p = \frac{(n+1)^{p+1}}{p+1} + \sum_{k=1}^p\frac{B_k}{p-k+1}{p\choose k}(n+1)^{p-k+1}$

$r=e-v+k+1$

Teorema de Pick: (Area, puntos interiores y puntos en el borde)

$A=I+\frac{B}{2}-1$


}%
\subsection{Ec. Caracteristica}
$a_0T(n)+a_1T(n-1)+...+a_kT(n-k)=0$

$p(x)=a_0 x^k + a_1 x^{k-1} + ... + a_k$

Sean $r_1,r_2,...,r_q$ las raíces distintas, de mult. $m_1, m_2, ..., m_q$

$T(n)=\sum_{i=1}^q{\sum_{j=0}^{m_i - 1}c_{ij} n^j r_i^n}$

Las constantes $c_{ij}$ se determinan por los casos base.
\subsection{Combinatorio}
\cppfile{math/combinatorio.cpp}
\subsection{Exp. de Numeros Mod.}
\cppfile[2]{math/exp.mod.cpp}
\subsection{Exp. de Matrices}
\cppfile{math/exp.mat.cpp}
\subsection{Matrices y determinante $O(n^3)$}
\cppfile[17-52]{math/determinante.cpp}
\subsection{Teorema Chino del Resto}
$$y=\sum_{j=1}^n (x_j*(\prod_{i=1, i\neq j}^n m_i)_{m_j}^{-1}*\prod_{i=1, i\neq j}^n m_i)$$
\subsection{Criba}
\cppfile[19-34
]{math/criba.cpp}
\subsection{Funciones de primos}
Sea $n=\prod{p_i^{k_i}}$, fact(n) genera un map donde a cada $p_i$ le asocia su $k_i$
\cppfile[33-88]{math/func.primos.cpp}
\subsection{Phollard's Rho (rolando)}
\cppfile{math/phollards.rho.villa.cpp}
\subsection{GCD}
\begin{code}
tipo gcd(tipo a, tipo b){return a?gcd(b %a, a):b;}
\end{code}
\subsection{Extended Euclid}
\cppfile{math/extended.euclid.cpp}
\subsection{LCM}
\begin{code}
tipo lcm(tipo a, tipo b){return a / gcd(a,b) * b;}
\end{code}
\subsection{Inversos}
\cppfile[7-16]{math/inversos.cpp}
\subsection{Simpson}
\cppfile{math/simpson.cpp}
\subsection{Fraction}
\cppfile{math/frac.cpp}
\subsection{Polinomio}
\cppfile[16-74]{math/polinomio.cpp}
\subsection{Ec. Lineales}
\cppfile[29-62]{math/eclineales.cpp}
\subsection{FFT}
\cppfile[16-65]{math/fft.cpp}
\subsection{Tablas y cotas (Primos, Divisores, Factoriales, etc)}
%\subsubsection{
 
\paragraph{Cantidad de primos menores que $10^n$}\ \\
$\pi(10^1)$ = 4 ;
$\pi(10^2)$ = 25 ;
$\pi(10^3)$ = 168 ;
$\pi(10^4)$ = 1229 ;
$\pi(10^5)$ = 9592 \\
$\pi(10^6)$ = 78.498 ;
$\pi(10^7)$ = 664.579 ;
$\pi(10^8)$ = 5.761.455 ;
$\pi(10^9)$ = 50.847.534 \\
$\pi(10^{10})$ = 455.052,511 ;
$\pi(10^{11})$ = 4.118.054.813 ;
$\pi(10^{12})$ = 37.607.912.018% ;
%
% Fuente: http://primes.utm.edu/howmany.shtml#table
%
%

%\subsubsection{Divisores}
\paragraph{Divisores} \ \\
Cantidad de divisores ($\sigma_0$) para \emph{algunos} $n / \neg\exists n'<n, \sigma_0(n') \geqslant \sigma_0(n)$ \\
$\sigma_0(60)$ = 12 ; $\sigma_0(120)$ = 16 ; $\sigma_0(180)$ = 18 ; $\sigma_0(240)$ = 20 ; $\sigma_0(360)$ = 24 \\
$\sigma_0(720)$ = 30 ; $\sigma_0(840)$ = 32 ; $\sigma_0(1260)$ = 36 ; $\sigma_0(1680)$ = 40 ; $\sigma_0(10080)$ = 72 \\ $\sigma_0(15120)$ = 80 ; $\sigma_0(50400)$ = 108 ; $\sigma_0(83160)$ = 128 ; $\sigma_0(110880)$ = 144 \\
$\sigma_0(498960)$ = 200 ; $\sigma_0(554400)$ = 216 ; $\sigma_0(1081080)$ = 256 ; $\sigma_0(1441440)$ = 288  $\sigma_0(4324320)$ = 384 ; $\sigma_0(8648640)$ = 448

\section{Grafos}%%%%%%%%%%%%%%%%%%GRAFOS%%%%%%%%%%%%%%%%%%%%%%%%%%%%
\subsection{Dijkstra}
\cppfile[6-23]{grafos/dijkstra.cpp}
\subsection{Bellman-Ford}
\cppfile{grafos/bellman.ford.cpp}
\subsection{Floyd-Warshall}
\cppfile{grafos/floyd.warshall.cpp}
\subsection{Kruskal}
\cppfile[27-41]{grafos/kruskal.cpp}
\subsection{Prim}
\cppfile[23-40]{grafos/prim.cpp}
\subsection{2-SAT + Tarjan SCC}
\cppfile{grafos/2sat.cpp}
\subsection{Articulation Points}
\cppfile{grafos/articulaciones.cpp}
\subsection{Comp. Biconexas y Puentes}
\cppfile[28-76]{grafos/biconexas.bridge.cpp}
\subsection{LCA + Climb}
\cppfile{grafos/lca.climb.cpp}
\subsection{Heavy Light Decomposition}
\cppfile[21-56]{grafos/heavylight.cpp}
\subsection{Centroid Decomposition}
\cppfile[17-36]{grafos/centroid.cpp}
\subsection{Euler Cycle}
\cppfile{grafos/euler.cpp}
\subsection{Diametro árbol}
\cppfile[17-41]{grafos/diametro.cpp}
\subsection{Chu-liu}
\cppfile{grafos/chuliu.villa.cpp}
\subsection{Hungarian}
\cppfile{grafos/hungarian.villa.cpp}
\subsection{Dynamic Conectivity}
\cppfile[17-78]{grafos/dynamic.conectivity.cpp}


\section{Network Flow}%%%%%%%%%%%%%%%%%%FLOW%%%%%%%%%%%%%%%%%%%%%%%%%%%%
\subsection{Dinic}
\cppfile[12-66]{flow/dinic.cpp}
%\subsection{Konig}
%\cppfile[98-122]{flow/konig.cpp}
%\subsection{Edmonds Karp's}
%\cppfile{flow/edmonds.karps.cpp}
%\subsection{Push-Relabel O(N3)}
%\cppfile{flow/push.relabel.cpp}
\subsection{Min-cost Max-flow}
\cppfile[16-66]{flow/min.cost.max.flow.cpp}


\section{Template}%%%%%%%%%%%%%%%%%%TEMPLATE%%%%%%%%%%%%%%%%
\cppfile{template.cpp}


\section{Ayudamemoria}%%%%%%%%%%%%%%%%%%AYUDAMEMORIA%%%%%%%%%%%%%%%%
\subsection*{Leer hasta fin de linea}
\begin{code}
#include <sstream>
//hacer cin.ignore() antes de getline()
while(getline(cin, line)){
   	 istringstream is(line);
   	 while(is >> X)
   		 cout << X << " ";
   	 cout << endl;
}
\end{code}
\subsection*{Expandir pila}
\begin{code}
#include <sys/resource.h>
rlimit rl;
getrlimit(RLIMIT_STACK, &rl);
rl.rlim_cur=1024L*1024L*256L;//256mb
setrlimit(RLIMIT_STACK, &rl);
\end{code}
\subsection*{Iterar subconjunto}
\begin{code}
for(int sbm=bm; sbm; sbm=(sbm-1)&bm)
\end{code}
\end{document}
